\section{POSIX Shared Memory}

POSIX shared memory API includes a set of functions that you can use
to create and use a shared memory region (segment) among processes.

Note that even though all processes are in memory and sharing the RAM
chip, they normally sit in non-overlapping regions of physical
memory. This is because each process has its own address space and the
address spaces of processes are mapped to different regions of
physical memory. When a shared segment is created, however, the
processes can share a portion of their address spaces and in this way
can all access the portion of physical memory.

We start by showing a simple two-process application. One process
creates shared memory segment of size 512 bytes and puts 100 small
integers there (each integer is 1 byte). The other process attaches to
the shared memory segment and reads those integers and prints them
out. To keep the programs short, we are not doing any error checking,
which must be done in a robust implementation.

Below is the code of the process putting integers to the shared
segment.

\VerbatimInput{posixsmem/simple_p.c}

Below is the code of the process retrieving the integers from the
shared segment and printing them out.

\VerbatimInput{posixsmem/simple_c.c}

Below is a Makefile to compile them. The same Makefile is used to
compile the next application we show. Note the we use the -lrt
compiler option to link the programs with the real-time (rt) library
that implements the shared memory API for Linux.

\VerbatimInput{posixsmem/Makefile}

Next, we provide the implementation of the classical producer-consumer
application using POSIX shared memory.  Producer and consumer
processes share a memory segment in which a buffer and three variables
are sitting: in, out, and count.

This solution is not free of race conditions. It can have race
conditions due to simultaneous access to shared data by more than one
process. But we are just ignoring this possibility at the moment. Most
of the time the program will run correctly.

In this example, we need to run the producer process first, since it
creates a shared segment. Then the consumer is run. Producer will
start generating and sending items (integers in this case) through the
shared segment to the consumer process. The consumer process will
retrieve the items from the shared memory and print them out.

Below is the header file {\tt commondefs.h} that includes the common
definitions used by the producer program and consumer program.

\VerbatimInput{posixsmem/commondefs.h}

In the header file, we define a structure in which we define the
variables (fields) that will be sitting in the shared memory
segment. Those fields are: a buffer and the variables in and out.  In
our program, we will define a pointer variable pointing to this
structure and that pointer will be initialized to point to the
beginning of the shared segment created. In this way, we will easily
access the content sitting in the shared memory by easily accessing to
the fields of the structure. The {\tt in} field will point to the
place (i.e., stores the index of the array entry) where the next
produced item will be put into. The {\tt out} field points to the
place in the array from where the next item will be retrieved by the
consumer process. The {\tt buffer} field of the structure will keep
the items.

Below is the producer program ({\tt producer.c}).

\VerbatimInput{posixsmem/producer.c} 

In the producer program, we create a shared memory segment using the
{\tt shm\_open} function (system call). We provide a name for the
segment to be created. The name we use is defined in the common header
file.  The shm\_open function returns a file descriptor to the calling
program. In subsequent related calls we use this file descriptor.  We
set the size of the shared segment using the {\tt ftruncate}
function. The same function is used to set the size of a file to some
value.  In this case, we set the size to the values {\tt
  SHAREDMEM\_SIZE} defined in our common header file.

Then, the {\tt mmap} function is used to obtain a pointer to the
beginning of the shared segment. In other words, we use mmap to map
the shared segment into the address space of the producer program. In
this way, the producer sees it as part of its address space and
accesses it by using the returned pointer value.  The pointer value is
stored in a local variable called {\tt shm\_start} in the producer
program. We can now use this pointer to access the shared memory
region. The shm\_start is a void pointer. We define another pointer
({\tt sdata\_ptr} which is of type {\tt struct shared\_data} and
therefore can be used to point to such a structure.  We initialize the
{\tt sdata\_ptr} pointer to point to the shared segment.

\begin{Verbatim}  
sdata\_ptr = (struct shared\_data *) shm\_start;
\end{Verbatim} 

In this way, we can then access the shared segment by using the
pointer and the fields of the structure it is pointing to. So, the
access to shared memory becomes so easy afterwards. We can say, for
example:

\begin{Verbatim}
        sdata_ptr->in = 0;
        sdata_ptr->out = 0;
\end{Verbatim} 

With this, we are initializing the part of the shared segment where
the variables (fields of the structure) in and out are located. Hence
we are accessing the shared memory by using ordinary memory access
(pointers and assignment statement). No system calls are done to
access the shared memory segment.

In the while loop, the producer program produces integers and puts
them into the buffer in the shared memory using ordinary memory
accesses.  As we said, no system calls are needed during this time.

Below is the consumer program ({\tt consumer.c}).

\VerbatimInput{posixsmem/consumer.c}

The consumer program opens the shared memory segment by using again
the {\tt shm\_open} system call. Then it uses mmap to map the stared
memory into its address space (i.e., gets a pointer to point to the
beginning of the shared memory). The pointer is stored in the local
variable {\tt shm\_start}. Another pointer (sdata\_ptr) of type {\tt
  struct shared\_data} is initialized to this pointer. Hence
sdata\_pointer can be now be used to access the shared memory in an
easy manner.

Below is a Makefile that can be used to compile the programs.

\VerbatimInput{posixsmem/Makefile}

We type {\tt make} to compile the programs and obtain two executable
files: producer and consumer. We run the producer first since it
creates the shared memory (consumer could create the shared memory as
well, instead of producer). Then we run the consumer. We can see what
is going on in the screen.

